%%%%%%%%%%%%%%%%%%%%%%%%%%%%%%%%%%%%%%%%%%%%%%%%%%%%%%%%
% 							                   PREAMBULE        
%%%%%%%%%%%%%%%%%%%%%%%%%%%%%%%%%%%%%%%%%%%%%%%%%%%%%%%%

\documentclass[a4,12pt]{article}

%--- Packages génériques ---%

\usepackage[francais]{babel}
\usepackage[utf8]{inputenc}
\usepackage[T1]{fontenc}
\usepackage[babel=true]{csquotes}
\usepackage{amsmath}
\usepackage{amssymb}
\usepackage{float}
\usepackage{graphicx}
\usepackage{hyperref}
\usepackage{algorithm}
\usepackage{algorithmicx}
\usepackage{algpseudocode}

\graphicspath{{res/}}

%--- Structure de la page ---%

\usepackage{fancyheadings}

\topmargin -1.5 cm
\oddsidemargin -0.5 cm
\evensidemargin -0.5 cm
\textwidth 17 cm
\setlength{\headwidth}{\textwidth}
\textheight 24 cm
\pagestyle{fancy}
\lhead[\fancyplain{}{\thepage}]{\fancyplain{}{\sl Mini project}}
\rhead[\fancyplain{}{}]{\fancyplain{}{Camille Gras \& Quentin Romero \& Benjamin Roussillon}}
\lfoot{\fancyplain{}{}}
\cfoot{\fancyplain{}{}}
\cfoot{\thepage }
\rfoot{\fancyplain{}{}}

%--- Style de la zone de code ---%

\usepackage{tikz}
\usetikzlibrary{calc}
\usepackage[framemethod=tikz]{mdframed}
\usepackage{listings}             
\usepackage{textcomp}

\lstset{upquote=true,
        columns=flexible,
        keepspaces=true,
        breaklines,
        breakindent=0pt,
        basicstyle=\ttfamily,
        commentstyle=\color[rgb]{0,0.6,0},
        language=Scilab,
        alsoletter=\),
        }

\lstset{classoffset=0,
        keywordstyle=\color{violet!75},
        deletekeywords={zeros,disp},
        classoffset=1,
        keywordstyle=\color{cyan},
        morekeywords={zeros,disp},
        }

\lstset{extendedchars=true,
        literate={0}{{\color{brown!75}0}}1 
                 {1}{{\color{brown!75}1}}1 
                 {2}{{\color{brown!75}2}}1 
                 {3}{{\color{brown!75}3}}1 
                 {4}{{\color{brown!75}4}}1 
                 {5}{{\color{brown!75}5}}1 
                 {6}{{\color{brown!75}6}}1 
                 {7}{{\color{brown!75}7}}1 
                 {8}{{\color{brown!75}8}}1 
                 {9}{{\color{brown!75}9}}1 
                 {(}{{\color{blue!50}(}}1 
                 {)}{{\color{blue!50})}}1 
                 {[}{{\color{blue!50}[}}1 
                 {]}{{\color{blue!50}]}}1
                 {-}{{\color{gray}-}}1
                 {+}{{\color{gray}+}}1
                 {=}{{\color{gray}=}}1
                 {:}{{\color{orange!50!yellow}:}}1
                 {é}{{\'e}}1 
                 {è}{{\`e}}1 
                 {à}{{\`a}}1 
                 {ç}{{\c{c}}}1 
                 {œ}{{\oe}}1 
                 {ù}{{\`u}}1
                 {É}{{\'E}}1 
                 {È}{{\`E}}1 
                 {À}{{\`A}}1 
                 {Ç}{{\c{C}}}1 
                 {Œ}{{\OE}}1 
                 {Ê}{{\^E}}1
                 {ê}{{\^e}}1 
                 {î}{{\^i}}1 
                 {ô}{{\^o}}1 
                 {û}{{\^u}}1 
        }

%--- Raccourcis commande ---%

\newcommand{\R}{\mathbb{R}}
\newcommand{\N}{\mathbb{N}}
\newcommand{\Z}{\mathbb{Z}}
\newcommand{\A}{\mathbf{A}}
\newcommand{\B}{\mathbf{B}}
\newcommand{\C}{\mathbf{C}}
\newcommand{\D}{\mathbf{D}}
\newcommand{\ub}{\mathbf{u}}
\newcommand{\wb}{\mathbf{w}}
\DeclareMathOperator{\e}{e}

%--- Mode correction et incréments automatiques ---%

\usepackage{framed}
\usepackage{ifthen}
\usepackage{comment}

\newcounter{Nbquestion}

\newcommand*\question{%
\stepcounter{Nbquestion}%
\textbf{Question \theNbquestion. }}

\newboolean{enseignant}
%\setboolean{enseignant}{true}
\setboolean{enseignant}{false}

\definecolor{shadecolor}{gray}{0.80}

\ifthenelse{
\boolean{enseignant}}{
\newenvironment{correction}{\begin{shaded}}{\end{shaded}}
}
{
\excludecomment{correction}
}

%--- Style de l'encadré des réponses ---%

\mdfsetup{leftmargin=12pt}
\mdfsetup{skipabove=0.0em,skipbelow=0.0em}

\tikzset{
	warningsymbol/.style={
	rectangle,draw=blue,
	fill=white,scale=1,
	overlay}}
\global\mdfdefinestyle{exampledefault}{
	hidealllines=true,leftline=true,
	innerrightmargin=0.0em,
	innerleftmargin=0.3em,
	leftmargin=0.0em,
	linecolor=blue!50,
	backgroundcolor=blue!10,
	middlelinewidth=4pt,
	%innertopmargin=\topskip,
}

%%%%%%%%%%%%%%%%%%%%%%%%%%%%%%%%%%%%%%%%%%%%%%%%%%%%%%%%
% 							               EN-TETE        
%%%%%%%%%%%%%%%%%%%%%%%%%%%%%%%%%%%%%%%%%%%%%%%%%%%%%%%%

\title{\textbf{Mini project \\ Are you the one ? }}
\author{
\begin{tabular}{cc}
	\textsc{Camille Gras, } \textsc{Quentin Romero \&} \textsc{Benjamin Roussillon} 
\end{tabular}}   
\date{\small \today}

\makeatletter
	\def\thetitle{\@title}
	\def\theautaur{\@autaur}
	\def\thedate{\@date}
\makeatother 

\usepackage{etoolbox}
\usepackage{titling}
\setlength{\droptitle}{-7em}

\setlength{\parindent}{0cm}

\makeatletter
% patch pour le bug concernant les parenthèses fermantes d'après http://tex.stackexchange.com/q/69472
\patchcmd{\lsthk@SelectCharTable}{%
  \lst@ifbreaklines\lst@Def{`)}{\lst@breakProcessOther)}\fi}{}{}{}
  
%%%%%%%%%%%%%%%%%%%%%%%%%%%%%%%%%%%%%%%%%%%%%%%%%%%%%%%%
% 							CORPS DU DOCUMENT          
%%%%%%%%%%%%%%%%%%%%%%%%%%%%%%%%%%%%%%%%%%%%%%%%%%%%%%%%

\begin{document}
\maketitle

\begin{center}
	\includegraphics[width=0.8\textwidth]{AreYouTheOne.jpg} 
\end{center}
\newpage

\section*{Presentation of the problem}

\begin{mdframed}[style=exampledefault]
10 women and 10 men have been paired by psychologists and an algorithm into ten couples. Their goal is to find their "perfect matches". Indeed, if they find the 10 perfect matches, they all share \$1 million. If they don't find all the matches, no one earn any money.
That's why we can see this game as a two player's game: the candidates versus the production. \vspace{2mm}

To find their soul mate, the twenty singles, which all live in a huge villa, have dates. \newline
Every five days, they propose ten couples (with distinct persons) and they know how many couples of their matching are right but not which ones. They have ten such trials called "matching ceremony". If they find the solution, the game is over and they all go home. If they didn't find the perfect matching after 10 trials, they lose the money. \vspace{2mm}

But that's not all. Before every matching ceremony, the candidates choose one couple among the three pairs which won the "date challenge" and test it in the "truth booth". It allows them to know if the tested couple is a correct match. If it is the case, they go to the honeymoon suite and will automatically be paired up for the remainder of the matching ceremonies. \vspace{2mm}

As the candidates, aged between 20 and 30, are paid to live in a beautiful villa, they don't want to win too early. Furthermore, most of them don't want to go to the "truth booth" with their perfect match as they don't want to go to the honeymoon suite. That's why the couples they make for the date challenges are strongly biased. \newline
Another reason which makes the candidates choices biased is that some of them mix up their feelings and the game. //Est ce qu'on le modelisera ? 

\end{mdframed}
 
\section*{Model}
\subsection*{The problem}
\begin{mdframed}[style=exampledefault]
We can denote that the probability that the boy $i$ matches with the girl $j$: $P_{ij}$. \newline
Therefore, we can build a matrix $P$ containing these probabilities.
\[P = \begin{bmatrix}
\dfrac{1}{10} & \hdots & \dfrac{1}{10} \\ 
\vdots & \ddots & \vdots \\ 
\dfrac{1}{10} & \cdots & \dfrac{1}{10}  \\
\end{bmatrix}
\]
//on peut penser à un graphe avec des probabilités sur les arêtes et on modifierait les arêtes au court du temps.\newline
$\bold{Remark}$ \newline
The probability to find the correct matching at the first trial is of:\newline
$P_{first Trial} =\dfrac{1}{10}\times\dfrac{1}{9}\times\dfrac{1}{8}\times\hdots\times\dfrac{1}{2} = \dfrac{1}{10 !} \simeq \dfrac{1}{3628800}$

\end{mdframed}

\subsection*{The matching ceremony}
//Je ne sais pas comment représenter l'info k couples sont justes parmi les n=10 couples proposés. \newline
//J'ai vaguement envie de faire un truc logique avec des ou

\subsection*{The "truth booth"}
\subsubsection*{This is a match !}
\begin{mdframed}[style=exampledefault] 
If we were testing the boy $i$ with the girl $j$: $P_{ij} = 1$ \newline
And $P_{ik} = 0 $ $ \forall k \neq j$ and $P_{kj} = 0 $ $ \forall k \neq i$\newline
The problem is reduced to the size $n-1$.
\end{mdframed}
\subsubsection*{This is not a match...}
\begin{mdframed}[style=exampledefault] 
If we were testing the boy $i$ with the girl $j$: $P_{ij} = 0$ \newline
and $\forall k \neq j$  $P_{ik} = \left\{ 
	\begin{array}{ll}
	0 &$ if $P_{ik}$ was 0$ \\
	\dfrac{1}{\frac{1}{P_{ik}}-1} &$ otherwise $ \end{array}\right.$
\newline
and $\forall k \neq i$  $P_{kj} = \left\{ 
	\begin{array}{ll}
	0 &$ if $P_{kj}$ was 0$ \\
	\dfrac{1}{\frac{1}{P_{kj}}-1} &$ otherwise $ \end{array}\right.$\newline
\end{mdframed}

\section*{Simulation}

\section*{Numerical results}

\end{document}
